\documentclass[a4paper,12pt]{article}
\usepackage[T1]{fontenc}
\PassOptionsToPackage{defaults=hu-min}{magyar.ldf}
\usepackage[magyar]{babel}


\begin{document}
	
	\begin{table}[ht!]
		\centering
	\begin{tabular}{|r|r|r|r|r|}
		\multicolumn{1}{c}{}&\multicolumn{4}{c}{év}\\
		\cline{2-5}
		\multicolumn{1}{c}{}&
		\multicolumn{1}{c}{\emph{2008}}&	
		\multicolumn{1}{c}{\emph{2009}}&
		\multicolumn{1}{c}{\emph{2010}}&
		\multicolumn{1}{c}{\emph{2011}}\\
		\hline
		\multicolumn{1}{|1|}{\emph{jövedelem (Ft)}}& 994\,000 & 1\,231\,500 & 1\,525\,410 & 2\,321\,600\\
		\multicolumn{1}{|1|}{\emph{járulék (Ft)}}& 994\,000 & 1\,231\,500 & 1\,525\,410 & 2\,321\,600\\
		\hline
	\end{tabular}
	\caption{A jövedelem és a járulékok kimutatása}
	\label{tablazat-jovedelem}
	\end{table}
	\Az{\ref{tablazat-jovedelem}}. ~táblázat\footnote{text} természetesen kitalált értéket tartalmaz, csak a gyakorlás kedvéért kellenek.
	
	
\end{document}
