\documentclass[a4paper,12pt]{article}
\usepackage[T1]{fontenc}
\PassOptionsToPackage{defaults=hu-min}{magyar.ldf}
\usepackage[magyar]{babel}
\begin{document}
	A halmazelmélet fiatal tudományág, létrejötte a XIX.~század második
	felére tehető, ami nem véletlen, hiszen a halmazok vizsgálatához nagyfokú
	absztrakció szükséges. Ekkorra értek el a matematikai kutatások olyan szintet, hogy az ilyen absztrakció szükségessé és lehetővé vált. Az előzmények
	közül a következő három a legfontosabb.
	\begin{enumerate}
		\item A matematikusok figyelme a halmazok elemeiről a halmazokra irányult.
		Olyan problémák vezettek ide, melyeket bizonyos halmazokra anélkül
		sikerült megoldani, hogy azokat az egyes halmazelemekre vonatkoztatták volna (pl.~biztosítási matematika, kinetikus gázelmélet).
%		\begin{enumerate}
%			\item al lista 1
%			\begin{enumerate}
%				\item al al lista 1
%				\begin{enumerate}
%					\item al al al lista 1
%				\end{enumerate}
%			\end{enumerate}
%		\end{enumerate}
		\item A kritikai szellem fejlődése, ami azt jelentette, hogy részletesen elemezték a korábban magától értetődőnek és ezért általános érvényűnek
		tekintett megállapításokat. (Ennek nagy szerepe volt a matematikai
		logika fejlődésében is.)
		\item A legdöntőbb momentum az volt, amikor a végtelen sorok vizsgálata
		közben felismerték, hogy a véges halmazok tulajdonságaival nem rendelkeznek törvényszerűen a \emph{végtelen halmazok} is.
	\end{enumerate}

	A ma \emph{naiv halmazelméletnek} nevezett rendszer megalkotója \textsc{Georg Cantor} (1845--1918) volt, akitől a halmaz fogalmának az alábbi körülírása származik: ,,A halmaz meghatározott, különböző, képzeletünkben vagy gondolatainkban fölfogott dolgok összessége. A kérdéses dolgok a halmaz elemei\dots''
	
	A továbbiakban az alapvető halmazelméleti fogalmakat -- részhalmaz, halmazok egyenlősége, műveletek, számosság stb.~-- ismertnek tételezzük fel, hiszen az analízis tárgyalásakor ezeket az olvasó megismerte.
	
	\textsc{Cantor} vizsgálta először a halmazelmélet egyik alapvető problémáját, az
	ún.~\emph{kontinuumhipotézist}, amely szerint nem létezik olyan halmaz, amelynek
	számossága a megszámlálhatóan végtelen halmazok számosságánál nagyobb,
	de a kontinuumszámosságnál kisebb.
	
	\bigskip
	A végtelen halmazok elméletének kezdettől fogva voltak bírálói, de addig
	szilárd elmélet volt, míg logikai ellentmondásokat nem fedeztek fel benne.
	Egyike ezeknek a Russell-féle antinómia, melynek népszerű változata a következő: \emph{A falu borbélya az a férfi a faluban, aki azokat és csak azokat a
		férfiakat borotválja meg a faluban, akik nem maguk borotválkoznak. Kérdés,
		hogy borotválkozik-e a borbély?}
	
	\begin{flushright}
	{\footnotesize 	\emph{,,A jó matematikus nem halmozza az élvezeteket,\\
	hanem élvezi a halmazokat.''}}
	\end{flushright}
\end{document}
