\documentclass[12pt, twoside]{report}
\usepackage{fancyhdr}
\usepackage[
colorlinks,
linktocpage,
allcolors=blue,
bookmarksnumbered
]{hyperref}
\usepackage[
a4paper, 
top = 40mm,
bottom = 40mm,
inner = 40mm,
outer = 30mm,
headsep = 8mm,
headheight = 15pt
]{geometry}
\usepackage[T1]{fontenc}
\PassOptionsToPackage{defaults=hu-min}{magyar.ldf}
\usepackage[magyar]{babel}

\usepackage{hulipsum}

\pagestyle{fancy}
\fancyhf{}
\fancyhead[EL, OR]{\thepage}
\fancyhead[ER]{\nouppercase{\small\sffamily\leftmark}}
\fancyhead[OL]{\nouppercase{\small\sffamily\rightmark}}
\begin{document}
	\title{Példa a report dokumentumosztály használatára}
	\author{Szerző}
	\date{évszám}
	\maketitle
	
	\tableofcontents
	\chapter*{Bevezetés}
	\markboth[Bevezetés]{Bevezetés}
	\hulipsum
	
	\chapter{Fejezet címe}\label{chapter-valoszinuseg}
	\section{Alfejezet címe}
	
	\Aref{chapter-valoszinuseg}.~fejezetben
	\hulipsum
	
	(Lásd \cite{DENKINGER}.)
	
	(Lásd \cite[15.~oldal]{DENKINGER}.)
	
	(Lásd \cite{DAROCZY, FAZEKAS}.)
	
	\begin{thebibliography}{4}
		\bibitem{DAROCZY}
		Daróczy Zoltán: Mérték és integrál, Budapest, 1984, Tankönyvkiadó.
		\bibitem{DENKINGER}
		Denkinger Géza: Valószínűség számítási gyakorlatok, Budapest, 1986, Tankönyvkiadó
		\bibitem{FAZEKAS}
		Fazekas István: Valószínűségszámítás, Debrecen, 2000, Debreceni Egyetem Kossuth Egyetemi Kiadója.
		\bibitem{SOLT}
		Solt György: Valószínűségszámítás, Budapest, 1993, Műszaki Könyvkiadó.
	\end{thebibliography}
\end{document}
