\documentclass[a4paper,12pt]{article}
\usepackage[T1]{fontenc}
\PassOptionsToPackage{defaults=hu-min}{magyar.ldf}
\usepackage[magyar]{babel}
\usepackage{graphicx, url}
\footnotestyle{rule=fourth}

\begin{document}
	\Aref{fig-lion}.~ábrán látható oroszlánt \textsc{Duane Bibby} -- neves amerikai grafikus --
	tervezte. A kép innen letölthető:
	\begin{center}
		\url{https://tomacstibor.uni-eszterhazy.hu/tananyagok/lion.pdf}
	\end{center}
	Ez az 5\,cm széles kép ma már a \LaTeX, pontosabban a \LaTeXe\ szimbólumává
	is vált
	\begin{figure}[ht!]
		\centering
		\includegraphics[width=5cm]{lion}
		\caption{A \TeX\ szimbóluma}
		\label{fig-lion}
	\end{figure}	
	
	\begin{figure}[ht!]
		\centering
		\includegraphics[width=5cm,angle=90,origin=c]{lion}%
		\includegraphics[width=5cm,angle=-90,origin=c]{lion}
		\caption{A \TeX\ szimbólum elforgatva}
		\label{fig-lion-forgatva}
	\end{figure}
	
	\Apageref{fig-lion-forgatva}.~oldalon látható \ref{fig-lion-forgatva}.~ábrát úgy kaptuk, hogy \aref{fig-lion}.~ábrát 90 illetve $-90$
	fokkal elforgattuk a középpontja körül.\footnote{Az oldalon található hivatkozásokat automatikus kereszthivatkozásként illesszük a
		dokumentumba!}
	
\end{document}
